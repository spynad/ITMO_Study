\begin{frame}{Измерение количества информации}
\noindent\color[rgb]{0,0.5,0.0}\textbf{Количество информации $\equiv$ информационная энтропия  - }
\color{black}
это численная мера
непредсказуемости информации. Количество информации в некотором объекте

\noindent определяется непредсказуемостью состояния, в котором находится этот объект.

\noindent Пусть i (s) — функция для измерения количеств информации в объекте s, состоящем из n
независимых частей $s_k$
, где k изменяется от 1 до n. Тогда 
\color[rgb]{0,0.5,0.0}\textbf{свойства меры количества информации}
\color{black}
\textbf{i(s)} таковы:

\begin{itemize}
	\item[\textbullet] Неотрицательность: i(s) $\geq$ 0.
	\item[\textbullet] Принцип предопределённости: если об объекте уже все известно, то i(s) = 0.
	\item[\textbullet] Аддитивность: i(s) = $\sum i(s_k)$ по всем k.
	\item[\textbullet] Монотонность: i(s) монотонна при монотонном изменении вероятностей.
\end{itemize}

\end{frame}