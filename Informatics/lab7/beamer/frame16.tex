\begin{frame}{Пример применения меры Хартли на практике}
\parindent=25mm
\noindent\textbf{Пример 1.}
Ведущий загадывает число от 1 до 64. Какое количество вопросов типа
«да-нет» понадобится, чтобы гарантировано угадать число?



\begin{itemize}
	\item[\textbullet] \underline{Первый} вопрос: «Загаданное число меньше 32?». Ответ: «Да».
	\item[\textbullet] \underline{Второй} вопрос: «Загаданное число меньше 16?». Ответ: «Нет».
	\item[\textbullet] ...
	\item[\textbullet] \underline{Шестой} вопрос (в худшем случае) точно приведёт к верному ответу
	\item[\textbullet] Значит, в соответствии с мерой Хартли в загадке ведущего содержится ровно $log_2 64 = 6$ бит непредсказуемости (т. е. информации).
\end{itemize}

\noindent\textbf{Пример 2.}
Ведущий держит за спиной ферзя и собирается поставить его на
произвольную клетку доски. Насколько непредсказуемо его решение?
\begin{itemize}
	\item[\textbullet] Всего на доске 8х8 клеток, а цвет ферзя может быть белым или чёрным, т. е.
	всего возможно 8х8х2 = 128 равновероятных состояний. 

	\item[\textbullet] Значит, количество информации по Хартли равно
	$log_2 128 = 7$ бит
\end{itemize}

\end{frame}