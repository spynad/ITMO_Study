\begin{frame}{Определение термина "информатика"}
\noindent\color[rgb]{0,0.7,0.4}\textbf{Информатика}
\color{black}
- дисциплина, изучающая свойста и структуру информации, закономерности ее создания,
преобразования, накопления, передачи и использования.

\noindent\color[rgb]{0,0.7,0.4}\textbf{Англ: }
\color{black}
informatics = information technology + computer science + information theory
\vspace{1.0em}
\begin{center}
	\textbf{Важные даты}
\end{center}
\begin{itemize}
	\item[\textbullet] 1956 – появление термина «информатика» (нем. Informatik, Штейнбух)
	\item[\textbullet] 1968 – первое упоминание в СССР (информология, Харкевич)
	\item[\textbullet] 197Х – информатика стала отдельной наукой
	\item[\textbullet] 4 декабря – день российской информатики
\end{itemize}
\end{frame}